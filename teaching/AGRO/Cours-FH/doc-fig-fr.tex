\documentclass[a4paper, french]{article}

\usepackage[french]{babel}
\usepackage[utf8]{inputenc}
\usepackage[T1]{fontenc}   		% A utiliser pour la mise en page finale et les guillemets
\usepackage{lmodern}          % fortement conseillé pour les pdf. On peut mettre autre chose : kpfonts, fourier,...

\usepackage{graphicx}


\title{Le titre en français}
\author{A. Author}
\date{\today}

\begin{document}

\maketitle

% ------------------------------------------------------------------------------------------------
\begin{abstract}
Ici, c'est la place du résumé
\end{abstract}
% ------------------------------------------------------------------------------------------------

% ------------------------------------------------------------------------------------------------
% ------------------------------------------------------------------------------------------------
\section{Introduction}
\label{sec:intro}
% ------------------------------------------------------------------------------------------------
% ------------------------------------------------------------------------------------------------

Le problème \ldots

% ------------------------------------------------------------------------------------------------
% ------------------------------------------------------------------------------------------------
\section{Méthode}
\label{sec:methode}
% ------------------------------------------------------------------------------------------------
% ------------------------------------------------------------------------------------------------

Notre démarche\footnote{Inspirée par celle d'Einstein en 1905.} \ldots telle que décrite dans la section \ref{sec:intro}.

% ------------------------------------------------------------------------------------------------
\subsection{Préparation de l'échantillon}
% ------------------------------------------------------------------------------------------------

% ------------------------------------------------------------------------------------------------
\subsection{Collecte des données}
% ------------------------------------------------------------------------------------------------

% ------------------------------------------------------------------------------------------------
% ------------------------------------------------------------------------------------------------
\section{Résultats}
% ------------------------------------------------------------------------------------------------
% ------------------------------------------------------------------------------------------------

\begin{figure}
\centering
\includegraphics[width=0.35\linewidth,angle=35]{Fig-roc-curves.pdf}
\caption{En rouge : la courbe ROC de deux fonctions de base $f_i$ and $f_j$. En bleu : la courbe ROC des fonctions  $\frac{|\cap_n^{i,j}|}{n}$ quand $n$ varie de 0 à $m$. }
\label{fig-roc-curves}
\end{figure}

\begin{figure}
\centering
\includegraphics[width=0.35\linewidth]{Fig-roc-curves.pdf}
\includegraphics[width=0.35\linewidth,angle=25]{Fig-roc-curves.pdf}
\caption{En rouge : la courbe ROC de deux fonctions de base $f_i$ and $f_j$. En bleu : la courbe ROC des fonctions  $\frac{|\cap_n^{i,j}|}{n}$ quand $n$ varie de 0 à $m$. }
\label{fig-roc-curves}
\end{figure}



% ------------------------------------------------------------------------------------------------
% ------------------------------------------------------------------------------------------------
\section{Conclusion}
% ------------------------------------------------------------------------------------------------
% ------------------------------------------------------------------------------------------------

\end{document}